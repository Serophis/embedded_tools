\newpage
\subsection{Fonctions principales développées} %2.3 Fonctions principales développées

Ce chapitre présente les fonctionnalités principales développées dans les incréments 1 et 2 en utilisant
une démarche par cas d'usage et par Cas d’Utilisation (CU).

\subsubsection{Rappel sur les cas d'usage}\label{rappelCasUsage}
%2.3.1 Rappel sur les cas d'usage

Les cas d'usage recensent les étapes essentielles du cycle de vie d'un produit depuis sa fabrication 
jusqu'à l'élimination ou le recyclage de ce produit.
À chaque étape du cycle de vie correspond un cas d'usage (si cette étape induit des
fonctionnalités à définir pour le produit considéré). Pour chaque cas d'usage, plusieurs cas
d'utilisation distincts peuvent être définis. \\

Parmi les cas d'usage, on retrouve généralement ceux de fabrication du produit (comprenant ou
non les activités de test du produit fabriqué), de conditionnement (paramétrage éventuel,
expédition et transport), de commercialisation (correspondant par exemple à un paramétrage éventuel, un mode de démonstration
ou une installation sur site), d'utilisation (par le ou les utilisateurs), de maintenance (SAV ou
diagnostic), de désinstallation et de recyclage (élimination ou valorisation).

\subsubsection{Rappel sur les cas d'utilisation} %2.3.2 Rappel sur les cas d'utilisation

Un Cas d’Utilisation (CU) représente un ensemble d’interactions entre un ou des acteurs et le
système à spécifier. Un cas d'utilisation étant souvent lié à un ou parfois plusieurs cas d'usage.
Un CU est principalement décrit par un scénario d’utilisation (nommé scénario nominal), scénario
d’une utilisation représentative du système. Ces CU sont ensuite détaillés jusqu’à un niveau de
décomposition suffisant pour décrire les fonctions attendues du système.

\paragraph{Représentation graphique des CU} %2.3.2.1 Représentation graphique des CU

Les CU peuvent être représentés sous forme graphique telle que décrite par la Figure \ref{fig:explication_diagramme_cas_utilisation}. Les acteurs directs sont représentés sous forme de petits personnages. Dans les bulles sont représentés les cas d’utilisation. Un trait entre un acteur et un CU indique que l’acteur participe à ce CU. Les liens hachurés entre CU, préfixés par le mot <<include>>, indiquent que ce CU fait appel à d’autres CU. On parle alors de sous-cas d’utilisation. Les liens hachurés entre CU, préfixés par le mot <<extend>>, indiquent qu’il s’agit d’une extension d’un CU, c'est à dire un CU qui ne se déclenche que sous certaines conditions.

\begin{figure}[H]
    \centering
    \includegraphics[width=10cm]{explication_diagramme_cas_utilisation}
    \caption{Légende explicative d'un diagramme de cas d'utilisation}
    \label{fig:explication_diagramme_cas_utilisation}
\end{figure}

\bigskip
\paragraph{Représentation textuelle des CU} %2.3.2.2 Représentation textuelle des CU

La description textuelle des cas d’utilisation est souvent présentée sous forme d’un tableau constitué des champs suivants :
\begin{xltabular}{\linewidth}{|l|X|}

    \hline
    Titre & Rappelle, en quelques mots, l’objectif principal du cas d’utilisation \\ \hline
    Résumé & Décrit brièvement le comportement du cas d’utilisation. \\ \hline
    Portée & Définit la portée de conception du CU (étendue spatiale). \\ \hline
    Niveau & Niveau de granularité du cas d’utilisation (Stratégique, utilisateur ou fonction). \\ \hline
    Acteurs directs & Acteurs qui participent au CU. \\ \hline
    Acteurs indirects & Acteurs qui ne participent pas au CU, mais qui ont un intérêt dans sa réalisation. \\ \hline
    Préconditions & Ensemble des conditions qui doivent être vérifiées avant le déroulement du CU. Les préconditions, sans mention contraire explicite, des CU parents au CU doivent
    toujours être vérifiées. \\ \hline
    Garanties minimales & Définissent ce qui est garanti par le système à l’étude même en cas d’échec du cas d’utilisation. \\ \hline
    Garanties en cas de succès & Définissent ce qui est garanti par le système à l'étude en cas de succès du cas d'utilisation (par le scénario nominal ou par ses variantes). \\ \hline

    Scénario nominal & C’est un scénario représentatif de l’utilisation du système où tout se passe bien. Il se termine par la réussite des objectifs. Il peut être constitué d’une condition déclenchant le scénario, d’un ensemble d’étape, d’une condition de fin, et éventuellement
    ses variantes. Une étape peut être une interaction entre acteurs, une
    étape de validation, ou un changement interne. \\ \hline

    Variantes & Cheminement différent du scénario nominal sans pour autant compromettre sa réussite. \\ \hline

    Extensions & Définissent les autres scénarios que le scénario nominal (par exemple ceux qui se
    terminent par un échec). Elles se déclenchent sur des conditions spécifiques détectées par le SaE. \\ \hline

    Informations complémentaires & Informations diverses nécessaires à la compréhension du CU. \\ \hline

\end{xltabular}

\subsubsection{Résumé des cas d'usage considérés} \label{section:resume_cas_d'usage}%2.3.3 Résumé des cas d'usage considérés

\complete Résumer les cas d'usage considérés dans ce document :
\begin{itemize}
    \item \complete Utilisation...
    \item \complete Maintenance...
    \item \complete Debug...
    \item \complete ...
\end{itemize}

\newpage
\newcommand{\CUstrategique}{\complete Commande CUstrategique Nom du CU stratégique}

\subsubsection{CU -- \CUstrategique}
\label{sec:cu_strategique}
\paragraph{Description graphique}

\begin{figure}[H]
    \centering
    \includegraphics[width=12cm]{cu_strategique}
    \caption{CU \CUstrategique}
    \label{fig:cu_strategique}
\end{figure}

\paragraph{Description textuelle}
La sous-section courante présente la description textuelle sous forme de tableau de la Figure \ref{fig:cu_strategique}.

\begin{xltabular}{\linewidth}{|l|X|}                                                                           \hline
    Titre                               & \CUstrategique                                                    \\ \hline
    Résumé                              & \complete                                                         \\ \hline
    Portée                              & \gls{sae}                                                         \\ \hline
    Niveau                              & Stratégique                                                       \\ \hline
    Acteurs directs                     & \complete                                                         \\ \hline
    Acteurs indirects                   & \complete                                                         \\ \hline
    Préconditions                       & \complete                                                         \\ \hline
    Garanties minimales                 & \complete ou aucune, à voir                                       \\ \hline
    Garanties en cas de succès          & \complete                                                         \\ \hline

    Scénario nominal                    &\begin{enumerate}
        \item \complete Nom de l'auteur | verbe conjugé | action
    \end{enumerate}                                                                                         \\ \hline

    Variantes                           & \begin{enumerate}
        \item[X-X :] [Expression du besoin/service de la variante]
            \begin{enumerate}
                \item [X-X] \complete Étapes de la variante
            \end{enumerate}
    \end{enumerate}                                                                                         \\ \hline

    Extensions                 & \gls{na}                                                                   \\ \hline
\end{xltabular}


\newpage
\newcommand{\CUexemple}{\complete Commande CUexemple Nom du CU stratégique}

\subsubsection{CU -- \CUexemple}
\label{sec:cu_exemple}
\paragraph{Description graphique}
\gls{na}
\paragraph{Description textuelle}

\begin{xltabular}{\linewidth}{|l|X|}                                                                           \hline
    Titre                               & \CUexemple                                                        \\ \hline
    Résumé                              & \complete                                                         \\ \hline
    Portée                              & \complete                                                         \\ \hline
    Niveau                              & \complete                                                         \\ \hline
    Acteurs directs                     & \complete                                                         \\ \hline
    Acteurs indirects                   & \complete                                                         \\ \hline
    Préconditions                       & \complete                                                         \\ \hline
    Garanties minimales                 & \complete ou aucune, à voir                                       \\ \hline
    Garanties en cas de succès          & \complete                                                         \\ \hline

    Scénario nominal                    &\begin{enumerate}
        \item \complete Nom de l'auteur | verbe conjugé | action
    \end{enumerate}                                                                                         \\ \hline

    Variantes                           & \begin{enumerate}
        \item[X-X :] [Expression du besoin/service de la variante]
            \begin{enumerate}
                \item [X-X] \complete Étapes de la variante
            \end{enumerate}
    \end{enumerate}                                                                                         \\ \hline

    Extensions                 & \gls{na}                                                                   \\ \hline
\end{xltabular}
