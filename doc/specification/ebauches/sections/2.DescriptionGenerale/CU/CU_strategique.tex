\newcommand{\CUstrategique}{\complete Commande CUstrategique Nom du CU stratégique}

\subsubsection{CU -- \CUstrategique}
\label{sec:cu_strategique}
\paragraph{Description graphique}

\begin{figure}[H]
    \centering
    \includegraphics[width=12cm]{cu_strategique}
    \caption{CU \CUstrategique}
    \label{fig:cu_strategique}
\end{figure}

\paragraph{Description textuelle}
La sous-section courante présente la description textuelle sous forme de tableau de la Figure \ref{fig:cu_strategique}.

\begin{xltabular}{\linewidth}{|l|X|}                                                                           \hline
    Titre                               & \CUstrategique                                                    \\ \hline
    Résumé                              & \complete                                                         \\ \hline
    Portée                              & \gls{sae}                                                         \\ \hline
    Niveau                              & Stratégique                                                       \\ \hline
    Acteurs directs                     & \complete                                                         \\ \hline
    Acteurs indirects                   & \complete                                                         \\ \hline
    Préconditions                       & \complete                                                         \\ \hline
    Garanties minimales                 & \complete ou aucune, à voir                                       \\ \hline
    Garanties en cas de succès          & \complete                                                         \\ \hline

    Scénario nominal                    &\begin{enumerate}
        \item \complete Nom de l'auteur | verbe conjugé | action
    \end{enumerate}                                                                                         \\ \hline

    Variantes                           & \begin{enumerate}
        \item[X-X :] [Expression du besoin/service de la variante]
            \begin{enumerate}
                \item [X-X] \complete Étapes de la variante
            \end{enumerate}
    \end{enumerate}                                                                                         \\ \hline

    Extensions                 & \gls{na}                                                                   \\ \hline
\end{xltabular}
