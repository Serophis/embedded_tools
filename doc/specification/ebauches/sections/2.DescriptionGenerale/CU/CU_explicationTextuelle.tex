\begin{xltabular}{\linewidth}{|l|X|}

    \hline
    Titre & Rappelle, en quelques mots, l’objectif principal du cas d’utilisation \\ \hline
    Résumé & Décrit brièvement le comportement du cas d’utilisation. \\ \hline
    Portée & Définit la portée de conception du CU (étendue spatiale). \\ \hline
    Niveau & Niveau de granularité du cas d’utilisation (Stratégique, utilisateur ou fonction). \\ \hline
    Acteurs directs & Acteurs qui participent au CU. \\ \hline
    Acteurs indirects & Acteurs qui ne participent pas au CU, mais qui ont un intérêt dans sa réalisation. \\ \hline
    Préconditions & Ensemble des conditions qui doivent être vérifiées avant le déroulement du CU. Les préconditions, sans mention contraire explicite, des CU parents au CU doivent
    toujours être vérifiées. \\ \hline
    Garanties minimales & Définissent ce qui est garanti par le système à l’étude même en cas d’échec du cas d’utilisation. \\ \hline
    Garanties en cas de succès & Définissent ce qui est garanti par le système à l'étude en cas de succès du cas d'utilisation (par le scénario nominal ou par ses variantes). \\ \hline

    Scénario nominal & C’est un scénario représentatif de l’utilisation du système où tout se passe bien. Il se termine par la réussite des objectifs. Il peut être constitué d’une condition déclenchant le scénario, d’un ensemble d’étape, d’une condition de fin, et éventuellement
    ses variantes. Une étape peut être une interaction entre acteurs, une
    étape de validation, ou un changement interne. \\ \hline

    Variantes & Cheminement différent du scénario nominal sans pour autant compromettre sa réussite. \\ \hline

    Extensions & Définissent les autres scénarios que le scénario nominal (par exemple ceux qui se
    terminent par un échec). Elles se déclenchent sur des conditions spécifiques détectées par le SaE. \\ \hline

    Informations complémentaires & Informations diverses nécessaires à la compréhension du CU. \\ \hline

\end{xltabular}