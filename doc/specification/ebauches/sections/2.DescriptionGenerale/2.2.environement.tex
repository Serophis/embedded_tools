\subsection{Environnement} % 2.2 Environnement

\subsubsection{Architecture matérielle et logicielle} % 2.2.1 Architecture matérielle et logicielle 
\label{section:achitecture_materielle_logicielle}
Le diagramme de déploiement \gls{uml} de la Figure \ref{diagramme_deploiement} représente l’architecture matérielle et logicielle du \gls{sae}. Les conventions graphiques utilisées sont décrites sur la Figure \ref{explcation_diagramme_deploiement}. Ce type de diagramme identifie les entités matérielles et/ou logicielles avec lesquelles le \gls{sae} doit interagir et permet ainsi de déterminer les principaux échanges qu’il entretient avec son environnement.

\begin{figure}[H]
    \centering
    \includegraphics[width=13cm]{diagramme_deploiement}
    \caption{Architecture matérielle et logicielle du \gls{sae} représentée par un diagramme de déploiement \gls{uml}}
    \label{diagramme_deploiement}
\end{figure}

\begin{figure}[H]
    \centering
    \includegraphics[width=7cm]{explication_diagramme_deploiement}
    \caption{Légende du diagramme de déploiement \gls{uml}}
    \label{explcation_diagramme_deploiement}
\end{figure}

Comme indiqué sur la Figure \ref{diagramme_deploiement}, le \gls{sae} est en interaction avec :
\begin{itemize}
    \item \complete
\end{itemize}

\subsubsection{Interfaces du système} \label{section:interface_du_systeme}% 2.2.2 Interfaces du système

Ce chapitre décrit les entrées et sorties dites "logiques" et "physiques" du \gls{sae}. En effet, nous différencions dans cette étude deux grands types d'entrées/sorties :

\begin{itemize}
    \item celles de haut niveau (dites logiques), décrivent les événements et données échangés entre \complete et \complete et entre ...\complete. Ces entrées/sorties décrivent les intentions de \complete ... ;
    \item celles de bas niveau (dites physiques) sont les entrées/sorties réellement échangées entre \complete et \complete.
\end{itemize}

\paragraph{Interfaces logiques} % 2.2.2.1 Interfaces Logiques 
La Figure \ref{contexte_logiqueUML} présente le contexte de projet en faisant figurer les entrées/sorties dites de haut niveau (ou logiques). Pour représenter ce contexte logique, un diagramme de communication UML a été utilisé.

\begin{figure} [H]
    \centering
    \includegraphics[width=13cm]{context_diagram}
    \caption{Contexte logique représenté par un diagramme de communication UML}
    \label{contexte_logiqueUML}
\end{figure}

Dans ce diagramme, seules les entrées/sorties logiques entre \complete et \complete sont représentées. Les entrées/sorties logiques entre \complete et \complete sont décrites dans la section \ref{section:interface_du_systeme}.

\input{sections/2.DescriptionGenerale/2.2.2.2.interfaceAvecLesActeurs.tex} % Fichier autogénéré par generate_context.py

\paragraph{Interfaces physiques} % 2.2.2.3 Interfaces physiques

\gls{na}

\paragraph{Interfaces de communication} %2.2.2.4 Interfaces de communication

\gls{na}

\subsubsection{Contraintes de mémoire} %2.2.3 Contraintes de mémoires

\gls{na}

\subsubsection{Activités} %2.2.4 Activités

\gls{na}

\subsubsection{Exigences d'adaptation} %2.2.5 Exigences d'adaptation

\gls{na}