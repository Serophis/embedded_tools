% WARNING: automatically generated file that may be overwritten or removed at any time

\section{Conception générale}

\newcommand\macroSuffix{}
\input{../animUML/conceptionGenerale/conceptionGenerale-macros}


\subsection{Architecture logique}

\begin{figure}[H]
	\centering
	\includegraphics[scale=.8,max width=\textwidth,max height=.9\textheight]{../animUML/conceptionGenerale/conceptionGenerale-context}
	\caption{Architecture logique}
	\label{fig:conceptionGenerale-archiLogique}
\end{figure}
Le diagramme de la \autoref{fig:conceptionGenerale-archiLogique} représente l'architecture logique du système.
\input{sections/2.ConceptionGenerale/descriptionArchiLogique/description_archi_logique.tex}

\subsection{Grands principes de fonctionnement}
\input{sections/2.ConceptionGenerale/descriptionSeq/description_grands_princpes_fonctionnement.tex}

\subsubsection{CU Hello Alice et Bob}
\begin{figure}[H]
	\centering
	\includegraphics[scale=.8,max width=\textwidth,max height=.9\textheight]{../animUML/conceptionGenerale/conceptionGenerale-sequence-Hello_Alice_et_Bob}
	\caption{Diagramme de séquence \emph{Hello Alice et Bob}}
	\label{fig:conceptionGenerale-inter-Hello_Alice_et_Bob}
\end{figure}
La \autoref{fig:conceptionGenerale-inter-Hello_Alice_et_Bob} représente le diagramme de séquence \emph{Hello Alice et Bob} 
                qui décrit le cas d'utilisation \textbf{Hello Alice et Bob}.
\input{sections/2.ConceptionGenerale/descriptionSeq/description_Hello_Alice_et_Bob.tex}


\subsection{Types de données}
\subsubsection{Types énumérés}
\paragraph{Vue d'ensemble des littéraux d'énumération}

\begin{figure}[H]
	\centering
	\includegraphics[scale=.8,max width=\textwidth,max height=.9\textheight]{../animUML/conceptionGenerale/conceptionGenerale-datatypes}
	\caption{Diagramme d'ensemble des littéraux d'énumération}
	\label{fig:conceptionGenerale-datatypes}
\end{figure}
Le diagramme de la \autoref{fig:conceptionGenerale-datatypes} représente les littéraux d'énumérations utilisés. Ils sont décrits dans les sections suivantes.

\paragraph{L'énumération \emph{ScreenId}}
L'énumération ScreenId possède les littéraux suivants :
\enumScreenIdLiteralDescriptions
\input{sections/2.ConceptionGenerale/descriptionTypes/description_type_ScreenId.tex}

\subsubsection{Autres types de données}
\paragraph{Le type \emph{Step}}
\input{sections/2.ConceptionGenerale/descriptionTypes/description_type_Step.tex}

\paragraph{Le type \emph{Hour}}
\input{sections/2.ConceptionGenerale/descriptionTypes/description_type_Hour.tex}

\paragraph{Le type \emph{Minute}}
\input{sections/2.ConceptionGenerale/descriptionTypes/description_type_Minute.tex}

\paragraph{Le type \emph{Value}}
\input{sections/2.ConceptionGenerale/descriptionTypes/description_type_Value.tex}

\paragraph{Le type \emph{Power}}
\input{sections/2.ConceptionGenerale/descriptionTypes/description_type_Power.tex}

\paragraph{Le type \emph{Temperature}}
\input{sections/2.ConceptionGenerale/descriptionTypes/description_type_Temperature.tex}


\subsection{Classes}

\subsubsection{Vue générale}

\begin{figure}[H]
	\centering
	\includegraphics[scale=.8,max width=\textwidth,max height=.9\textheight]{../animUML/conceptionGenerale/conceptionGenerale-classes}
	\caption{Diagramme de classes}
	\label{fig:conceptionGenerale-classes}
\end{figure}
Le diagramme de la \autoref{fig:conceptionGenerale-classes} représente les classes du système.
\input{sections/2.ConceptionGenerale/descriptionClass/description_classes.tex}

\subsubsection{La classe Alice}

\begin{figure}[H]
	\centering
	\includegraphics[scale=.8,max width=\textwidth,max height=.9\textheight]{../animUML/conceptionGenerale/conceptionGenerale-class-Alice}
	\caption{Diagramme de la classe Alice}
	\label{fig:conceptionGenerale-class-Alice}
\end{figure}
Le diagramme de la \autoref{fig:conceptionGenerale-class-Alice} représente la classe Alice.
\input{sections/2.ConceptionGenerale/descriptionClass/description_Alice.tex}

\paragraph{Attributs}
\classAliceProperties
\paragraph{Services offerts}
\classAliceOperations
\paragraph{Description comportementale}
\begin{figure}[H]
	\centering
	\includegraphics[scale=.8,max width=\textwidth,max height=.9\textheight]{../animUML/conceptionGenerale/conceptionGenerale-Alice-SM}
	\caption{Machine à états de \emph{Alice}}
	\label{fig:conceptionGenerale-sm-Alice}
\end{figure}
Le diagramme de la \autoref{fig:conceptionGenerale-sm-Alice} représente la machine à états de \emph{Alice}.
\input{sections/2.ConceptionGenerale/descriptionMAE/description-MAE-Alice.tex}
\subsubsection{La classe Bob}

\begin{figure}[H]
	\centering
	\includegraphics[scale=.8,max width=\textwidth,max height=.9\textheight]{../animUML/conceptionGenerale/conceptionGenerale-class-Bob}
	\caption{Diagramme de la classe Bob}
	\label{fig:conceptionGenerale-class-Bob}
\end{figure}
Le diagramme de la \autoref{fig:conceptionGenerale-class-Bob} représente la classe Bob.
\input{sections/2.ConceptionGenerale/descriptionClass/description_Bob.tex}

\paragraph{Attributs}
\classBobProperties
\paragraph{Services offerts}
\classBobOperations
\paragraph{Description comportementale}
\begin{figure}[H]
	\centering
	\includegraphics[scale=.8,max width=\textwidth,max height=.9\textheight]{../animUML/conceptionGenerale/conceptionGenerale-Bob-SM}
	\caption{Machine à états de \emph{Bob}}
	\label{fig:conceptionGenerale-sm-Bob}
\end{figure}
Le diagramme de la \autoref{fig:conceptionGenerale-sm-Bob} représente la machine à états de \emph{Bob}.
\input{sections/2.ConceptionGenerale/descriptionMAE/description-MAE-Bob.tex}
